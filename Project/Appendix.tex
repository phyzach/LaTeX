\documentclass[12pt,oneside,a4paper,english]{article}
\usepackage[T1]{fontenc}
\usepackage[latin2]{inputenc}
\usepackage[margin=2.25cm,headheight=26pt,includeheadfoot]{geometry}
\usepackage[english]{babel}
\usepackage{listings}
\usepackage{color}
\usepackage{titlesec}
\usepackage{titling}
\usepackage[framed, numbered]{matlab-prettifier}
\usepackage{changepage}
\usepackage{amsmath}
\usepackage{hyperref}
\usepackage{enumitem}
\usepackage{graphicx}
\usepackage{fancyhdr}
\usepackage{lastpage}
\usepackage{caption}
\usepackage{tocloft}
\usepackage{setspace}
\usepackage{multirow}
\usepackage{titling}
\usepackage{float}
\usepackage{comment}
\usepackage{booktabs}
\usepackage{indentfirst}
\usepackage{lscape}
\usepackage{booktabs,caption}
\usepackage[flushleft]{threeparttable}
\usepackage[english]{nomencl}
\usepackage{xcolor}
\usepackage{lipsum}


% --- set footer and header ---
\pagestyle{fancy}
\fancyhf{}


% --- End of page settings ---
\begin{document}
\title{SIMBAD Alderbaran Page Definitions}
\maketitle

\pagenumbering{arabic} 
\section{SIMBAD Table Definitions}
We will include a glossay of terms with their reference in the \href{URL}{SIMBAD page}, the physical term, and the meaning of the term, this is based on the SIMBAD page for Alderbaran. The terms are organized by the section they are found in the SIMBAD page.
\begin{table}[H]
    \centering
    \caption{Summary of Terms and the Sections in a SIMBAD Star Page}
    \begin{tabular}{|c|p{3cm}|p{9cm}|}
    \centering
    \textbf{SIMBAD TERM} & \textbf{Physics Term} & \textbf{Definition} \\ \hline \hline
            ICRS coord  & Stellar Coordinates    & Reference location coordinate     \\ \hline   
            FK4 coord   & Stellar Coordinates    & Reference location coordinat(alternate coordinate system).  \\ \hline
            GAL coord   & Stellar Coordinates    & Reference location coordinat(alternate coordinate system).   \\ \hline
            Proper Motions(mas/yr) & Stellar Motion & Motion of star from earth in milliarcseconds per year.  \\ \hline 
            Radial velocity/Redshift/cz & Apparent velocity of star is moving from Earth & Velocity of star from earth in km/s, this is the doppler effect causing redshift.  \\ \hline
            Parallax(mas) & Parallax distance of the star in milliarcseconds & Using this value, the stellar distance to the star can be determined.  \\ \hline
            Spectral type & Stellar Class & The class of the star, refer to above for details. \\ \hline
            Fluxes & Luminosity & Luminosity of the star in various wavelengths starting from U - UV light to K - infared light \\ \hline
            Hierarchy & Objects in the stars orbit or vice versa & Can be planets, other stars, generally organizes with orbiting bodies being children of larger bodies. \\ \hline
            Identifier & Acronym Information & Due to the amount of data, there are many different nomenclature used, so clicking one of thes can give insight.  \\ \hline
            References & Academic Observations & Allows for historical review of past observational data and the read on the methodology used. \\ \hline
            Collections of Measurements & Measurements  & Direct link to measurements made by scientist with a reference to the academic article. See Section 5.  \\ \hline
            Observing logs & Observational Data & Link to raw observational data, this could be unformatted.  \\ \hline
            External Archives & External Databases & Links to other databases that may have more information on the star.  \\ \hline
    \end{tabular}
    \label{tab:table1}
\end{table}

    \begin{table}[H]
    \centering
    \caption{Velocity Table Definitions}
    \begin{tabular}{|c|p{3cm}|p{9cm}|}
    \centering
    \textbf{SIMBAD TERM} & \textbf{Phyiscs/Unit} & \textbf{Definition} \\ \hline \hline
    typ & Velocity type & Can be raw km/s(v), redshift(z), or a product of the speed of light and red shift(cz). Measure of Velocity from the earth, either as a velocity or a proportion of the redshift.  \\ \hline
    value & Velocity & The value of the velocity based on type above  \\ \hline
    R & Single character(?) & This indicates if there is potential systemic error in velocity.  \\ \hline
    m.e. & Error in Velocity & The error in the velocity measurement measured as a value of $\sigma$.  \\ \hline
    Acc & Letter Grade & Quality of measurement based on study, ranging from A-E.  \\ \hline
    Nmes & Number of Measurements & The number of measurements taken to determine the velocity.  \\ \hline
    nat & Measurement Type(p,s,se,sa) & Nature of measurement, refers to the sensor used to derive the velocity.\\ \hline
    Q & Letter Grade & The quality of the measurement, duplicate value as Acc.  \\ \hline
    dom & Domain & Wavelength domain of measurement(Radio,mm,Infared,Optical,UV,X-Ray,Gamma)  \\ \hline
    res & Resolution & Refers to error or resolution of the tool to measure the incoming light. \\ \hline
    D & Placeholder & Placeholder for data validation.  \\ \hline
    Obs.data & Julian Days & Date of observation given in Julian days \\ \hline
    Or & 2 Character code & Note of origin on the radial velocity measurement.  \\ \hline
    reference & Bibcode & Reference to the academic article where the data was published.  \\ \hline
    \end{tabular}
    \label{tab:table2}
\end{table}
\begin{table}[H]
    \centering
    \caption{Rotational Velocity Table Definitions}
    \begin{tabular}{|c|p{3cm}|p{9cm}|}
    \centering
    \textbf{SIMBAD TERM} & \textbf{Phyiscs/Unit} & \textbf{Definition} \\ \hline \hline
    upVsini & Boolean Flag & Flag to indicate if the rotational velocity is the upper limit(max values).  \\ \hline
    Vsini & Rotational Velocity(km/s) & The rotational velocity of the star in km/s.  \\ \hline
    err & Error in Rotational Velocity(km/s) & The error in the rotational velocity measurement.  \\ \hline
    Q & Letter Grade & The quality of the measurement(A-E)  \\ \hline
    reference & Bibcode & Reference to the academic article where the data was published.  \\ \hline
    \end{tabular}
    \label{tab:table3}
    \end{table}
\begin{table}[H]
    \centering
    \caption{Variability Table Definitions}
    \begin{tabular}{|c|p{3cm}|p{9cm}|}
    \centering
    \textbf{SIMBAD TERM} & \textbf{Phyiscs/Unit} & \textbf{Definition} \\ \hline \hline
    vartyp & Type of Variability & The type of variability the star exhibits(refer to \href{https://simbad.u-strasbg.fr/Pages/guide/chG.htx#v}{Variability Definitions}) \\ \hline
    Lomax & Single character & Flag to indicate the maximum brightness  \\ \hline
    max & $\frac{W}{m^2}$ & The maximum brightness of the star in watts per square meter.  \\ \hline
    R\_max & Single character & Flag to indicate the maximum brightness error.  \\ \hline
    band & Wavelength Band & The wavelength band of measurement(Radio,mm,Infared,Optical,UV,X-Ray,Gamma)\\ \hline
    Upmin & Single character & Flag to indicate the minimum brightness  \\ \hline
    UpPeriod & Single character & Flag to indicate the lower limit in a period.  \\ \hline
    period & Days & The period of the variability in Julian days.  \\ \hline
    R\_period & Single character & Flag to indicate the period error.  \\ \hline
    epoch & Julian Days & The time of variability measurement.  \\ \hline
    R\_epoch & Single character & Flag to indicate the epoch error.  \\ \hline
    D/rt & Julian Days & Special measurement for Algol type systems(special case). \\ \hline
    $\%$ & Single Character & Flag to indicate error or uncertainty on raising time.  \\ \hline
    reference & Bibcode & Reference to the academic article where the data was published.  \\ \hline
    \end{tabular}
    \label{tab:table3}
    \end{table}
\begin{table}[H]
    \centering
    \caption{Fe/H Table Definitions}
    \begin{tabular}{|c|p{3cm}|p{9cm}|}
    \centering
    \textbf{SIMBAD TERM} & \textbf{Phyiscs/Unit} & \textbf{Definition} \\ \hline \hline
    Teff & Kelvin & The effective temperature of the star in Kelvin.  \\ \hline
    log.g & Gravity $frac{m}{s^2}$ & Effective gravity of the star \\ \hline
    Fe\_H & Iron to Hydrogen Ratio & Also known as metallicity index in a log scale: -1 = 10x less metal than sun, +0.3 = 2x more metal than sun. Indicator for the age of the star.  \\ \hline
    c & Single Letter & Flag indicating the method of calculating the ratio. \\ \hline
    CompStar & Name or Identifier & Names a comparable star. \\ \hline
    CatNo & Number & Catalog number of the star in a certain study(Cayrel Et Al).  \\ \hline
    Reference & Bibcode & Reference to the academic article where the data was published.  \\ \hline
    \end{tabular}
    \label{tab:table4}
    \end{table}
    \begin{table}[H]
        \centering
        \caption{Parallax(Plx) Table Definitions}
        \begin{tabular}{|c|p{3cm}|p{9cm}|}
        \centering
        \textbf{SIMBAD TERM} & \textbf{Phyiscs/Unit} & \textbf{Definition} \\ \hline \hline
        plx & mas & Parallax distance of the star in milliarcseconds.  \\ \hline
        m.e & mas & Error in the parallax distance of the star in milliarcseconds.  \\ \hline
        R & 2 Character Code & Code of observatory that measured the parallax.  \\ \hline
        reference & Bibcode & Reference to the academic article where the data was published.  \\ \hline
        \end{tabular}
        \label{tab:table5}
    \end{table}
    \begin{table}[H]
        \centering
        \caption{Proper Motion(PM) Table Definitions}
        \begin{tabular}{|c|p{3cm}|p{9cm}|}
        \centering
        \textbf{SIMBAD TERM} & \textbf{Phyiscs/Unit} & \textbf{Definition} \\ \hline \hline
        pm-ra & mas/yr & Proper motion in right ascension in milliarcseconds per year.  \\ \hline
        m.e. & mas/yr & Error in the proper motion in right ascension in milliarcseconds per year.  \\ \hline
        pm-de & mas/yr & Proper motion in declination in milliarcseconds per year.  \\ \hline
        m.e. & mas/yr & Error in the proper motion in declination in milliarcseconds per year.  \\ \hline
        syst & FK4,FK5,ICRS Coordinates & Coordinate system used when measuring proper motion. \\ \hline
        reference & Bibcode & Reference to the academic article where the data was published.  \\ \hline
        \end{tabular}
        \label{tab:table6}
    \end{table}
    \begin{table}[H]
        \centering
        \caption{Spectral Type(SpT) Table Definitions}
        \begin{tabular}{|c|p{3cm}|p{9cm}|}
        \centering
        \textbf{SIMBAD TERM} & \textbf{Phyiscs/Unit} & \textbf{Definition} \\ \hline \hline
        ds/mss & single character with notes & Indicates the system used to make the classification.  \\ \hline
        SpType & Spectral Type & The spectral type of the star, refer to other tutorial for info  \\ \hline
        Reference & Bibcode & Reference to the academic article where the data was published.  \\ \hline
        \end{tabular}
        \label{tab:table7}
    \end{table}
    \section{Observational Logs}
    The observation logs are custodial and show us the raw data from the telescope, This data is organized chronologically and can be difficult to read. The data is in the form of a table including some of the following information:
    \begin{itemize}
        \item Size of region of space covered in arcmin.
        \item Observatory Code.
        \item Minimum and Maximum wavelength searching.
        \item Target the observatory was aiming for.
        \item Julian Date and time of the observation.
        \item Duration of the observation.
        \item FOV of the lens used.
        \item The steller coordinates of the telescope at the target.
    \end{itemize}
    \section{External Archives}
    The external archives are databases that contain more information on the star, these databases are not SIMBAD but are linked to SIMBAD. These databases can contain more information on the star and nearby objects around them, these should be avoided if you are not familiar with the data you're looking at.
\end{document}