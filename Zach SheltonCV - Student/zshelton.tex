%%%%%%%%%%%%%%%%%
% This is an example CV created using altacv.cls (v1.6.4, 13 Nov 2021) written by
% LianTze Lim (liantze@gmail.com), based on the
% Cv created by BusinessInsider at http://www.businessinsider.my/a-sample-resume-for-marissa-mayer-2016-7/?r=US&IR=T
%
%% It may be distributed and/or modified under the
%% conditions of the LaTeX Project Public License, either version 1.3
%% of this license or (at your option) any later version.
%% The latest version of this license is in
%%    http://www.latex-project.org/lppl.txt
%% and version 1.3 or later is part of all distributions of LaTeX
%% version 2003/12/01 or later.
%%%%%%%%%%%%%%%%

%% Use the "normalphoto" option if you want a normal photo instead of cropped to a circle
% \documentclass[10pt,a4paper,normalphoto]{altacv}

\documentclass[10pt,a4paper,ragged2e,withhyper]{altacv}

%% AltaCV uses the fontawesome5 package.
%% See http://texdoc.net/pkg/fontawesome5 for full list of symbols.

% Change the page layout if you need to
\geometry{left=1.25cm,right=1.25cm,top=1.5cm,bottom=1.5cm,columnsep=1.2cm}

% The paracol package lets you typeset columns of text in parallel
\usepackage{paracol}


% Change the font if you want to, depending on whether
% you're using pdflatex or xelatex/lualatex
\ifxetexorluatex
  % If using xelatex or lualatex:
  \setmainfont{Lato}
\else
  % If using pdflatex:
  \usepackage[default]{lato}
\fi

% Change the colours if you want to
\definecolor{SlateGrey}{HTML}{2E2E2E}
\definecolor{LightGrey}{HTML}{666666}
\definecolor{Salmon}{HTML}{FA8072}
\definecolor{Aquamarine}{HTML}{088F8F}
\definecolor{TealBlue}{HTML}{008080}
%\usepackage[dvipsnames]{xcolor}
\colorlet{heading}{TealBlue}
\colorlet{accent}{Salmon}
\colorlet{emphasis}{Aquamarine}
\colorlet{body}{LightGrey}

% Change some fonts, if necessary
% \renewcommand{\namefont}{\Huge\rmfamily\bfseries}
% \renewcommand{\personalinfofont}{\footnotesize}
% \renewcommand{\cvsectionfont}{\LARGE\rmfamily\bfseries}
% \renewcommand{\cvsubsectionfont}{\large\bfseries}

% Change the bullets for itemize and rating marker
% for \cvskill if you want to
\renewcommand{\itemmarker}{{\small\textbullet}}
\renewcommand{\ratingmarker}{\faCircle}

%% Use (and optionally edit if necessary) this .tex if you
%% want to use an author-year reference style like APA(6)
%% for your publication list
% When using APA6 if you need more author names to be listed
% because you're e.g. the 12th author, add apamaxprtauth=12
\usepackage[backend=biber,style=apa6,sorting=ydnt]{biblatex}
\defbibheading{pubtype}{\cvsubsection{#1}}
\renewcommand{\bibsetup}{\vspace*{-\baselineskip}}
\AtEveryBibitem{\makebox[\bibhang][l]{\itemmarker}}
\setlength{\bibitemsep}{0.25\baselineskip}
\setlength{\bibhang}{1.25em}


%% Use (and optionally edit if necessary) this .tex if you
%% want an originally numerical reference style like IEEE
%% for your publication list
% \usepackage[backend=biber,style=ieee,sorting=ydnt]{biblatex}
%% For removing numbering entirely when using a numeric style
\setlength{\bibhang}{1.25em}
\DeclareFieldFormat{labelnumberwidth}{\makebox[\bibhang][l]{\itemmarker}}
\setlength{\biblabelsep}{0pt}
\defbibheading{pubtype}{\cvsubsection{#1}}
\renewcommand{\bibsetup}{\vspace*{-\baselineskip}}


%% sample.bib contains your publications
\addbibresource{sample.bib}

\begin{document}
\name{ZACHARY SHELTON}
\tagline{\textcolor{Aquamarine}{Data Engineer | Physicist}}
% Cropped to square from https://en.wikipedia.org/wiki/Marissa_Mayer#/media/File:Marissa_Mayer_May_2014_(cropped).jpg, CC-BY 2.0
%% You can add multiple photos on the left or right
\photoR{3.5cm}{USEME.jpeg}
%\photoL{2cm}{Yacht_High,Suitcase_High}
\personalinfo{%
  % Not all of these are required!
  % You can add your own with \printinfo{symbol}{detail}
  \email{zacharshel10@gmail.com}
  \phone{+45 52 78 13 49}\\
  \location{Copenhagen, Demark}
    \linkedin{zshelton}
    \bigskip
    \\
   Data engineer and scientist with a strong desire to experience new and varied problems to further hone my skills in fields like climate study and geophysics. I hope to contribute in research and use my skills to create a brighter tomorrow.

   
%   
%   \orcid{0000-0000-0000-0000} % Obviously making this up too.
  %% You can add your own arbitrary detail with
  %% \printinfo{symbol}{detail}[optional hyperlink prefix]
  % \printinfo{\faPaw}{Hey ho!}
  %% Or you can declare your own field with
  %% \NewInfoFiled{fieldname}{symbol}[optional hyperlink prefix] and use it:
  % \NewInfoField{gitlab}{\faGitlab}[https://gitlab.com/]
  % \gitlab{your_id}
	%%
  %% For services and platforms like Mastodon where there isn't a
  %% straightforward relation between the user ID/nickname and the hyperlink,
  %% you can use \printinfo directly e.g.
  %% But if you absolutely want to create new dedicated info fields for
  %% such platforms, then use \NewInfoField* with a star:
  % \NewInfoField*{mastodon}{\faMastodon}
  %% then you can use \mastodon, with TWO arguments where the 2nd argument is
  %% the full hyperlink.
  % \mastodon{@username@instance}{https://instance.url/@username}
}
\textbf{AUTHORIZED TO WORK IN DENMARK}\\ 
\smallskip
\makecvheader


%% Depending on your tastes, you may want to make fonts of itemize environments slightly smaller
\AtBeginEnvironment{itemize}{\small}

%% Set the left/right column width ratio to 6:4.
\columnratio{0.6}

% Start a 2-column paracol. Both the left and right columns will automatically
% break across pages if things get too long.
\begin{paracol}{2}


\cvsection{Education}
\cvevent{MSc Physics, High Energy Physics - Non-thesis}{Florida Institute of Technology}{January 2020 - December 2021}{Melbourne, United States}



\textcolor{Aquamarine}{Using Machine Learning to identify Top Quarks in CMS (February 2021 - September 2021)}
\begin{itemize}
    \item Developed algorithms using the XGBoost Python library to identify and classify top quark pairs via their hadronic decay jets’ charge and geometric features. 
    \item Compared geometric and spatial orientation of 3 or more jets to identify most likely trio quarks from a hadronic top decay.
    \item Determined that boosted decision trees combined with data regression methods would be the most effective method for diverse datasets.
    \item Collaborated with peers from the University of Ohio to refine machine learning methodologies, integrating academic and applied approaches

\end{itemize}
\cvevent{BSc Physics}{Florida Institute of Technology}{August 2015 - December 2019}{Melbourne, United States}
\textcolor{Aquamarine}{Educational Cosmic Muon Detector - BSc Thesis (January 2019 - December 2019)}
\smallskip
\begin{itemize}
    \item Conducted spectrum analysis of scintillating materials sourced from FermiLab, evaluating their potential in educational applications. 
    \item Assembled and tested a simple cosmic ray detector for use in undergraduate modern physics courses made from a recycled scintillator. 
    \item Developed and presented lesson plans and lab materials aimed at introducing complex particle physics concepts to undergraduate and advanced high school students. 
    \item Presented findings at the Northrop Grumman Engineering and Science Fair, showcasing the intersection of research and education.

\end{itemize}
\cvevent{BSc STEM Education}{Florida Institute of Technology}{August 2015 - December 2019}{Melbourne, United States}
\textcolor{Aquamarine}{Apprentice Teaching (August 2019 - December 2019)}
\begin{itemize}
\item Taught four periods of secondary physics and mathematics. 
\item Scaffolded classroom activities for varied skill levels and understanding. 
\item Completed student evaluations and created testing material to track students’ progress throughout the semester. 
\item Adapted lesson planning according to state and national curriculum standards


\end{itemize}
\newpage
\cvsection{Work Experience}
\cvevent{Data Engineer}{Uptimus}{Aug 2024 - Dec 2024}{Copenhagen, DK}
\begin{itemize}
\item Leveraged scientific Python programming to design and implement scalable systems for data intake and analysis, processing large geospatial and temporal datasets for actionable insights.
\item Developed ETL pipelines in a CI/CD environment using Python, Azure, Docker, and pgAdmin4 to optimize real-time data processing.
\item Applied machine learning techniques, statistics, and probabilities to analyze fleet vehicle data, contributing to cost-saving initiatives in last-mile delivery.
\item Built data analysis tools for integration into mobile and web applications, enhancing customer value with accessible, data-driven insights.
\item Demonstrated scientific communication skills by delivering clear, data-focused reports and presentations to stakeholders, aligning data strategies with business needs.
\item Showcased proficiency in English by preparing high-quality documentation and participating in meetings with the Tech Lead, CEO, and stakeholders to support decision-making.
\item Authored detailed technical reports and explored opportunities for business growth through efficient data sourcing and stakeholder collaboration.
\item Adapted to a dynamic startup environment by independently setting KPIs, managing high-responsibility tasks, and achieving key objectives.

\end{itemize}

\cvevent{Student Research Coordinator}{Wolfram Research Inc.}{Jan 2020 - Feb 2024}{Copenhagen, DK (Remote)}
\begin{itemize}
    \item Managed and expanded the Wolfram Student Ambassador Program, mentoring ambassadors at universities worldwide and facilitating collaboration across multiple time zones.
    \item Advised students from diverse cultural and academic backgrounds on integrating Wolfram Language into academic projects, research initiatives, and data-driven problem-solving tasks, directly edited and contributed to academic articles and accompanying material.
    \item Trained students in building and debugging machine learning algorithms, developing neural networks, and performing feature extraction for research purposes.
    \item Explored and demonstrated Wolfram’s Quantum Framework, Neural Net Library, and advanced data analysis tools to identify innovative use cases and drive adoption.
    \item Actively communicated with international teams, including ambassadors, students, and professionals, across diverse time zones to ensure seamless coordination and program success.
    \item Conducted qualitative textual and sentiment analysis of Wolfram’s social media presence and media coverage, offering actionable insights for global engagement strategies. 
\end{itemize}

\cvevent{Undergraduate Research Assistant}{Florida Institute of Technology}{May 2016 - Dec 2019}{Melbourne, United States}
\begin{itemize}
    \item Designed Printed Circuit Boards (PCBs) using Eagle CAD for calibration of QIE charge injectors, simulating calorimeter responses for use in particle physics labs such as FermiLab and the CMS detector at CERN.
    \item Collaborated with a multinational team for three months at FermiLab, testing and troubleshooting CMS equipment to ensure operational accuracy and reliability.
    \item Maintained and improved the team’s Python software base while on-site at FermiLab, enhancing analysis efficiency and tool reliability.
    \item Utilized ROOT7 analysis framework and Python to process and analyze CERN Analysis Object Data (AOD), deriving insights critical to experimental validation.
    \item Worked under Associate Professor Dr. Francisco Yumiceva, contributing to internationally recognized particle physics research projects.

\end{itemize}



\switchcolumn

\cvsection{Computer languages \\\smallskip \& Techincal Skills}
\cvtag{Python}
\cvtag{PyTorch}
\cvtag{PostgreSQL}
\cvtag{TensorFlow}
\cvtag{Docker}
\cvtag{Pandas}
\cvtag{Deep Learning}\\
\cvtag{Neural Networks}
\cvtag{QGIS}
\cvtag{Keras}
\cvtag{XgBoost}
\cvtag{scikit}
\cvtag{Wolfram Language}
\cvtag{LaTeX}
\cvtag{HTML}
\cvtag{Git}
\cvtag{Numerical Analysis}\\
\cvtag{Bayesian Analysis}
\cvtag{ROOT7}
\cvtag{Azure}
\cvtag{Microsoft Excel, Word, PowerPoint}\\
\cvtag{C++}
\cvtag{FDTD Simulation}
\cvtag{PySpark}
\cvtag{Optics}\\
\cvtag{Experimental Design}
\cvtag{Web Scraping}\\
\cvtag{Electronic Measurement Techniques}\\
\cvtag{Digital Analog Signal Analysis}

\cvsection{Professional\\\smallskip Interests}

\cvtag{Data Science}
\cvtag{Climate Science}
\cvtag{Green Energy}\\
\cvtag{Particle Physics}
\cvtag{Experimental Physics}
\cvtag{Geophysics}
\cvtag{Green Technology}
\cvtag{DevOps}\\
\cvtag{Neural Networks}
\cvtag{AI}
\cvtag{Electrodynamics}
\cvtag{Scientific Literacy}\\






\medskip




\cvsection{Languages}
\cvtag{English(Native Speaker)} \\
\cvtag{Danish (Intermediate Proficiency)}

% \bigskip
% \bigskip
% \bigskip
% \bigskip
% \bigskip
% \bigskip
% \bigskip
% \bigskip
% \bigskip
% \bigskip
% \bigskip


\cvsection{Hobbies and \\\smallskip Interests}
\cvtag{Snowboarding}
\cvtag{Hiking}
\cvtag{Camping} \\
\cvtag{Gaming}
\cvtag{Electronics prototyping}


\newpage

%\cvsection{Relevant Coursework}
%\begin{itemize}
%    \item {\normalsize Data Analysis Methods}
%    \item {\normalsize Electromagnetic Theory 2}
%    \item {\normalsize Science and Technical Communications}
%    \item {\normalsize Mathematical Methods in Science and \\Engineering 1}
%    \item {\normalsize Quantum Mechanics 2}
%    \item {\normalsize Statistics in High Energy Physics}
%\end{itemize}
\end{paracol}

\end{document}
