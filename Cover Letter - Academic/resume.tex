%-------------------------
% Entry-level Cover-letter Template in LaTeX
% Made to go with "Entry-level Resume in laTeX" - here
% Version - v1.0
% Last Edits - October 5, 2021
% Author : Jayesh Sanwal
% Reach out to me on LinkedIn(/in/jsanwal), with any suggestions, ideas, issues, etc.
%------------------------


%%%%%%% --------------------------------------------------------------------------------------
%%%%%%%  STARTING HERE, DO NOT TOUCH ANYTHING 
%%%%%%% --------------------------------------------------------------------------------------

%%%% Define Document type
\documentclass[11pt,a4]{article}

%%%% Include Packages
\usepackage{latexsym}
\usepackage[empty]{fullpage}
\usepackage{titlesec}
 \usepackage{marvosym}
\usepackage[usenames,dvipsnames]{color}
\usepackage{verbatim}
\usepackage[hidelinks]{hyperref}
\usepackage{fancyhdr}
\usepackage{multicol}
\usepackage{hyperref}
\usepackage{csquotes}
\usepackage{tabularx}
\hypersetup{colorlinks=true,urlcolor=black}
\usepackage[11pt]{moresize}
\usepackage{setspace}
\usepackage{fontspec}
\usepackage[inline]{enumitem}
\usepackage{ragged2e}
\usepackage{anyfontsize}

%%%% Set Margins
\usepackage[margin=1cm]{geometry}

%%%% Set Fonts
\setmainfont[
BoldFont=SourceSansPro-Semibold.otf,
ItalicFont=SourceSansPro-RegularIt.otf
]{SourceSansPro-Regular.otf}

%%%% Set Page Style
\pagestyle{fancy}
\fancyhf{} 
\fancyfoot{}
\renewcommand{\headrulewidth}{0pt}
\renewcommand{\footrulewidth}{0pt}

%%%% Set URL Style
\urlstyle{same}

%%%% Set Indentation
\raggedbottom
\raggedright
\setlength{\tabcolsep}{0in}

%%%% Set Secondary Color, Page Number Color, Footer Text
\definecolor{UI_blue}{RGB}{32, 64, 151}
\definecolor{HF_color}{RGB}{179, 179, 179}

%%%% Set Heading Format
\titleformat{\section}{
\color{UI_blue} \scshape \raggedright \large 
}{}{0em}{}[\vspace{-0.7cm} \hrulefill \vspace{-0.2cm}]
%%%%%%% --------------------------------------------------------------------------------------
%%%%%%% --------------------------------------------------------------------------------------
%%%%%%%  END OF "DO NOT TOUCH" REGION
%%%%%%% --------------------------------------------------------------------------------------
%%%%%%% --------------------------------------------------------------------------------------



\begin{document}
%%%%%%% --------------------------------------------------------------------------------------
%%%%%%%  HEADER
%%%%%%% --------------------------------------------------------------------------------------
\begin{center}
    \begin{minipage}[b]{0.24\textwidth}
            \large 52 78 13 49 \\
            \large \href{mailto:zshelton1997@gmail.com}{zacharshel10@gmail.com} 
    \end{minipage}% 
    \begin{minipage}[b]{0.5\textwidth}
            \centering
            {\Huge Zach Shelton} \\ %
            \vspace{0.1cm}
            {\color{UI_blue} \Large{Physicist} \\
    \end{minipage}% 
    \begin{minipage}[b]{0.24\textwidth}
            \flushright \large
            {\href{https://www.linkedin.com/in/zshelton/}{linkedin.com/in/zshelton/} } \\

    \end{minipage}   
    
\vspace{-0.15cm} 
{\color{UI_blue} \hrulefill}
\end{center}

\justify
\setlength{\parindent}{0pt}
\setlength{\parskip}{12pt}
\vspace{0.1cm}

%%%%%%% --------------------------------------------------------------------------------------
%%%%%%%  First 2 Lines
%%%%%%% --------------------------------------------------------------------------------------

Date: \today \par \vspace{-0.1cm}

To whom it may concern,

I am writing to express my intent for an opportunity to grow my t




Through my academic journey, earning both my bachelor’s and master’s degrees, I have developed a strong foundation in the theoretical and statistical principles of high-energy physics. 
I joined a multi-national team at FermiLab in Chicago, where we tested and calibrated scintillating measurement devices for installation in the Compact Muon Solenoid (CMS) Experiment at CERN. These roles have given me the skills to join a lab setting and begin contributing quickly. During my studies, I also applied machine learning to analyze composite particles via their decay products, focusing on identifying top quark decay from reconstructed quark jets. To tackle this, I used deep learning neural networks and machine learning methods, such as boosted decision trees and regression modeling, leveraging Python packages like XgBoost, Keras, and TensorFlow. The challenge of handling vast datasets—ranging from 100,000 to over a billion particle events—ignited my interest in data science and motivated me to deepen my technical expertise in this area. Combined with my practical experience at FermiLab, I am prepared to apply my measurement techniques and
analysis in a research setting I’ve also been working at Wolfram Research assisting undergraduate and graduate students researching various projects and fields from PDE Solvers to analysis of high energy physics experiments. I’ve also become an proficent user of Wolfram Language and Mathematica. 

My masters was effected heavily by the pandemic, being moved to online courses for two semesters restricted my access to
research facilities. My advisor for my degree left the university, which left me with a consequential decision to make. I chose to earn a non-thesis masters in December of 2021 and move with my wife to Copenhagen. 

I have been working as a Data Scientist in Denmark building and growing my data science and programming skills learning important software and packages like Keras, Tensorflow, Git, Microsoft Azure and Docker. 
Across all of my roles, my skills with Python and essential libraries has deepened to point of personal pride, my problem solving process has improved to include numerical analysis and algorithm testing in Jupyter.

I want to utilize my background in physics in order to improve my career opportunities and contribute in fields that I find
interesting and important to me. I believe at KU, I will be able to grow my network and skills in order to create consequential research and contribute to new phyics and science!

Thank you for your time and your consideration. Please do not hesitate to contact me via phone or e-mail, I look forward to
hearing from you.
\par
%%%
%%%%%%% --------------------------------------------------------------------------------------
%%%%%%%  SIGNATURE
%%%%%%% --------------------------------------------------------------------------------------

\vspace{0.5cm}
\raggedright
Respectfully, \\ Zach Shelton \\ \\ +45 52 78 13 49 \\ 
\href{mailto:zshelton1997@gmail.com}{zacharshel10@gmail.com} 

\end{document}



%I am most proud of my work in Experimental Particle Physics with collision data from CERN. I participated in a multi-national team and create truly interesting and great work in machine learning in physics.

Secondly, I am glad I spent time teaching and advising students using Wolfram Language. The projects varied from introductory CS, to PhD research and big data projects.