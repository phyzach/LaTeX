%-------------------------
% Entry-level Cover-letter Template in LaTeX
% Made to go with "Entry-level Resume in laTeX" - here
% Version - v1.0
% Last Edits - October 5, 2021
% Author : Jayesh Sanwal
% Reach out to me on LinkedIn(/in/jsanwal), with any suggestions, ideas, issues, etc.
%------------------------
% !TEX program = xelatex

%%%% Define Document type
\documentclass[11pt,a4]{article}

%%%% Include Packages
\usepackage{latexsym}
\usepackage[empty]{fullpage}
\usepackage{titlesec}
 \usepackage{marvosym}
\usepackage[usenames,dvipsnames]{color}
\usepackage{verbatim}
\usepackage[hidelinks]{hyperref}
\usepackage{fancyhdr}
\usepackage{multicol}
\usepackage{hyperref}
\usepackage{csquotes}
\usepackage{tabularx}
\hypersetup{colorlinks=true,urlcolor=black}
\usepackage[11pt]{moresize}
\usepackage{setspace}
\usepackage{fontspec}
\usepackage[inline]{enumitem}
\usepackage{ragged2e}
\usepackage{anyfontsize}

%%%% Set Margins
\usepackage[margin=1cm]{geometry}

%%%% Set Fonts
\setmainfont[
BoldFont=SourceSansPro-Semibold.otf,
ItalicFont=SourceSansPro-RegularIt.otf
]{SourceSansPro-Regular.otf}

%%%% Set Page Style
\pagestyle{fancy}
\fancyhf{} 
\fancyfoot{}
\renewcommand{\headrulewidth}{0pt}
\renewcommand{\footrulewidth}{0pt}

%%%% Set URL Style
\urlstyle{same}

%%%% Set Indentation
\raggedbottom
\raggedright
\setlength{\tabcolsep}{0in}

%%%% Set Secondary Color, Page Number Color, Footer Text
\definecolor{UI_blue}{RGB}{32, 64, 151}
\definecolor{HF_color}{RGB}{179, 179, 179}

%%%% Set Heading Format
\titleformat{\section}{
\color{UI_blue} \scshape \raggedright \large 
}{}{0em}{}[\vspace{-0.7cm} \hrulefill \vspace{-0.2cm}]
%%%%%%% --------------------------------------------------------------------------------------
%%%%%%% --------------------------------------------------------------------------------------
%%%%%%%  END OF "DO NOT TOUCH" REGION
%%%%%%% --------------------------------------------------------------------------------------
%%%%%%% --------------------------------------------------------------------------------------



\begin{document}
%%%%%%% --------------------------------------------------------------------------------------
%%%%%%%  HEADER
%%%%%%% --------------------------------------------------------------------------------------
\begin{center}
    \begin{minipage}[b]{0.24\textwidth}
            \large 52 78 13 49 \\
            \large \href{mailto:zshelton1997@gmail.com}{zacharshel10@gmail.com} 
    \end{minipage}% 
    \begin{minipage}[b]{0.5\textwidth}
            \centering
            {\Huge Zach Shelton} \\ %
            \vspace{0.1cm}
            {\color{UI_blue} \Large{Data Engineer}} \\
    \end{minipage}% 
    \begin{minipage}[b]{0.24\textwidth}
            \flushright \large
            {\href{https://www.linkedin.com/in/zshelton/}{linkedin.com/in/zshelton/} } \\

    \end{minipage}   
    
\vspace{-0.15cm} 
{\color{UI_blue} \hrulefill}
\end{center}

\justify
\setlength{\parindent}{0pt}
\setlength{\parskip}{12pt}
\vspace{0.1cm}

%%%%%%% --------------------------------------------------------------------------------------
%%%%%%%  First 2 Lines
%%%%%%% --------------------------------------------------------------------------------------

Date: \today \par \vspace{-0.1cm}
%%%

Dear Bera Pálsdóttir,

Lightera is a company I have followed for some time, I explored career opportunities with Lightera last year. After meeting Tomas Youngman and Fouad Alassani at the DSE Career Fair in Lyngby, I am excited to see Lightera searching for new R\&D Engineers. As a physics student in both my bachelor's and graduate studies, I initially found optics and photonics to be my least favorite subject. I felt challenged and confused by optics and its intrinsic relationship to electromagnetism. However, as I progressed through graduate courses, I became fascinated by light-matter interactions and optical processes in different media.

I have grown as a data scientist, data engineer, and research advisor through my work at Wolfram Research and a Danish software startup. Yet I constantly reflect on my particle physics research, and I deeply desire to return to professional research. I am driven by the mission to test scientific theories and develop technologies that make a measurable positive impact on the world. At Lightera, I hope to find myself thinking, "I need to open a textbook or find an academic article to understand this better." Most importantly, I want to join a team of people who share my passion for physics and learning. Photonics engineering is increasingly fundamental to modern technologies, and I can see myself growing as both an engineer and researcher at Lightera.

When I began studying at Florida Institute of Technology, I joined my advisor Dr. Francisco Yumiceva's research group. As an undergraduate, I designed, prototyped, and implemented Charge Injector Boards (QIEs) to calibrate Silicon Photomultiplier (SiPM) detectors for the Compact Muon Solenoid (CMS) experiment at CERN. I transported these boards to Fermilab and collaborated with an international research team to define parameters and test the equipment before its installation at CERN. I also designed a scintillating material recycling system at Fermilab for use in student and enthusiast muon detectors. During my graduate studies, I worked with Dr. Yumiceva to develop machine learning methods for identifying hadronic boson decays using jet kinematics. Using Python and Linux HPCs, I simulated and analyzed data to train a boosted decision tree, collaborating with researchers from other universities to complete this work. This experience has prepared me to help plan tests and experiments, analyze results, and diagnose potential issues before they become problems.

At Wolfram Research, I taught the Wolfram Language and mentored university students on data-intensive projects spanning financial modeling, statistical analysis, and machine learning applications. These projects covered diverse fields including GIS, mathematics, physics, and chemistry, developing my ability to quickly learn new domains and communicate insights effectively to both technical and non-technical audiences. I managed students through regular meetings and by editing their work, while also writing and editing academic-style posts for Wolfram's Community forum. From this experience, I learned to manage multiple projects simultaneously while ensuring timely delivery of high-quality results. I am equally comfortable communicating with international technical and non-technical stakeholders, including customers.

Most recently, as a Data Engineer at Uptimus, a Danish startup focused on last-mile delivery optimization,I designed ETL pipelines and data integration solutions using Git. My work involved preparing production-ready scripts using tools like Docker, Azure, and PostgreSQL, along with performing geospatial analysis using QGIS and Python libraries. I grew significantly as a developer and data engineer, and I am prepared to learn new tools while transferring relevant techniques from my previous work to Lightera. I am highly proficient with Python, data visualization libraries like Matplotlib and Seaborn, as well as programs like Power BI.

Throughout my career, I have remained fascinated by real-world physics applications and modeling physical interactions. My inspiration to work in photonics came from a project where I modeled a Bragg reflector using FDTD methods in MATLAB. I am particularly interested in working with fibers for fiber amplifiers and lasers, though I would be happy to contribute to any of Lightera's projects. Simply working in a lab setting would fulfill a lifelong dream I made to myself as an undergraduate. I can provide references, including my direct supervisor at Uptimus, who can speak to my technical abilities and work ethic. Please feel free to contact me via phone or email to discuss my background or arrange an interview.
%%%
%%%%%%% --------------------------------------------------------------------------------------
%%%%%%%  SIGNATURE
%%%%%%% --------------------------------------------------------------------------------------

\vspace{0.5cm}
\raggedright
Respectfully, \\ Zach Shelton 
\\ 
+45 52 78 13 49 \\ 
\href{mailto:zacharshel10@gmail.com}{zacharshel10@gmail.com} 

\end{document}
