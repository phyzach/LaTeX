%-------------------------
% Entry-level Cover-letter Template in LaTeX
% Made to go with "Entry-level Resume in laTeX" - here
% Version - v1.0
% Last Edits - October 5, 2021
% Author : Jayesh Sanwal
% Reach out to me on LinkedIn(/in/jsanwal), with any suggestions, ideas, issues, etc.
%------------------------
% !TEX program = xelatex

%%%% Define Document type
\documentclass[11pt,a4]{article}

%%%% Include Packages
\usepackage{latexsym}
\usepackage[empty]{fullpage}
\usepackage{titlesec}
 \usepackage{marvosym}
\usepackage[usenames,dvipsnames]{color}
\usepackage{verbatim}
\usepackage[hidelinks]{hyperref}
\usepackage{fancyhdr}
\usepackage{multicol}
\usepackage{hyperref}
\usepackage{csquotes}
\usepackage{tabularx}
\hypersetup{colorlinks=true,urlcolor=black}
\usepackage[11pt]{moresize}
\usepackage{setspace}
\usepackage{fontspec}
\usepackage[inline]{enumitem}
\usepackage{ragged2e}
\usepackage{anyfontsize}

%%%% Set Margins
\usepackage[margin=1cm]{geometry}

%%%% Set Fonts
\setmainfont[
BoldFont=SourceSansPro-Semibold.otf,
ItalicFont=SourceSansPro-RegularIt.otf
]{SourceSansPro-Regular.otf}

%%%% Set Page Style
\pagestyle{fancy}
\fancyhf{} 
\fancyfoot{}
\renewcommand{\headrulewidth}{0pt}
\renewcommand{\footrulewidth}{0pt}

%%%% Set URL Style
\urlstyle{same}

%%%% Set Indentation
\raggedbottom
\raggedright
\setlength{\tabcolsep}{0in}

%%%% Set Secondary Color, Page Number Color, Footer Text
\definecolor{UI_blue}{RGB}{32, 64, 151}
\definecolor{HF_color}{RGB}{179, 179, 179}

%%%% Set Heading Format
\titleformat{\section}{
\color{UI_blue} \scshape \raggedright \large 
}{}{0em}{}[\vspace{-0.7cm} \hrulefill \vspace{-0.2cm}]
%%%%%%% --------------------------------------------------------------------------------------
%%%%%%% --------------------------------------------------------------------------------------
%%%%%%%  END OF "DO NOT TOUCH" REGION
%%%%%%% --------------------------------------------------------------------------------------
%%%%%%% --------------------------------------------------------------------------------------



\begin{document}
%%%%%%% --------------------------------------------------------------------------------------
%%%%%%%  HEADER
%%%%%%% --------------------------------------------------------------------------------------
\begin{center}
    \begin{minipage}[b]{0.24\textwidth}
            \large 52 78 13 49 \\
            \large \href{mailto:zshelton1997@gmail.com}{zacharshel10@gmail.com} 
    \end{minipage}% 
    \begin{minipage}[b]{0.5\textwidth}
            \centering
            {\Huge Zach Shelton} \\ %
            \vspace{0.1cm}
            {\color{UI_blue} \Large{Data Engineer}} \\
    \end{minipage}% 
    \begin{minipage}[b]{0.24\textwidth}
            \flushright \large
            {\href{https://www.linkedin.com/in/zshelton/}{linkedin.com/in/zshelton/} } \\

    \end{minipage}   
    
\vspace{-0.15cm} 
{\color{UI_blue} \hrulefill}
\end{center}

\justify
\setlength{\parindent}{0pt}
\setlength{\parskip}{12pt}
\vspace{0.1cm}

%%%%%%% --------------------------------------------------------------------------------------
%%%%%%%  First 2 Lines
%%%%%%% --------------------------------------------------------------------------------------

Date: \today \par \vspace{-0.1cm}
%%%

To whom it may concern,

Ørsted is a market leader in the green energy transition, and I hope to join as a sustainability data specialist. My experience in academic data analysis and professional data engineering, using tools such as Python, PostgreSQL, and other open-source data management solutions, has prepared me to contribute to Ørsted's market leadership.

During my studies, I collaborated with a multinational team on the Compact Muon Solenoid (CMS) Experiment at CERN, where I developed machine learning algorithms to analyze large-scale datasets. My work involved creating predictive models to isolate specific decay events from kinematics data using neural networks, decision trees, and libraries such as XGBoost, TensorFlow, and Keras. Additionally, I designed electronics and supporting software to test the functionality of PCBs, participating in the entire process of design, testing, and implementation. These experiences not only honed my Python skills but also strengthened my ability to manage complex data structures, perform exploratory analysis, and contribute to the broader ecosystem of physical and software systems that underpin modern data pipelines.

At Wolfram Research, I taught the Wolfram Language and guided university students on data-intensive projects, including financial modeling, statistical analysis, and machine learning applications. These projects spanned diverse fields such as GIS, mathematics, physics, and chemistry, equipping me with the adaptability to quickly learn new domains and communicate insights effectively to both technical and non-technical stakeholders. I managed these projects using a SCRUM-like approach, ensuring proper timelines and accountability. These experiences have positioned me well to contribute to software development initiatives.

Most recently, as a Data Engineer at Uptimus, a Danish startup focused on last-mile delivery optimization, I designed ETL pipelines and data integration solutions using GIT. My work involved preparing production-ready scripts, leveraging tools like Docker, Azure, and PostgreSQL, and applying geospatial analysis with QGIS and Python libraries. By working with data streamed from delivery vehicles across Sjælland, I developed the ability to think on my feet and continuously improve processes. Outside of my role at Uptimus, I have been exploring PowerBI and learning to effectively manage data within the Microsoft ecosystem.

At Ørsted, I look forward to contributing to dynamic data streaming pipelines while deepening my expertise in Python and exploring new techniques. 

%%%
%%%%%%% --------------------------------------------------------------------------------------
%%%%%%%  SIGNATURE
%%%%%%% --------------------------------------------------------------------------------------

\vspace{0.5cm}
\raggedright
Respectfully, \\ Zach Shelton 
\\ 
+45 52 78 13 49 \\ 
\href{mailto:zacharshel10@gmail.com}{zacharshel10@gmail.com} 

\end{document}
