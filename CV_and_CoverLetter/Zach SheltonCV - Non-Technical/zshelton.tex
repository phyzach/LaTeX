%%%%%%%%%%%%%%%%%
% This is an example CV created using altacv.cls (v1.6.4, 13 Nov 2021) written by
% LianTze Lim (liantze@gmail.com), based on the
% Cv created by BusinessInsider at http://www.businessinsider.my/a-sample-resume-for-marissa-mayer-2016-7/?r=US&IR=T
%
%% It may be distributed and/or modified under the
%% conditions of the LaTeX Project Public License, either version 1.3
%% of this license or (at your option) any later version.
%% The latest version of this license is in
%%    http://www.latex-project.org/lppl.txt
%% and version 1.3 or later is part of all distributions of LaTeX
%% version 2003/12/01 or later.
%%%%%%%%%%%%%%%%

%% Use the "normalphoto" option if you want a normal photo instead of cropped to a circle
% \documentclass[10pt,a4paper,normalphoto]{altacv}

\documentclass[10pt,a4paper,ragged2e,withhyper]{altacv}

%% AltaCV uses the fontawesome5 package.
%% See http://texdoc.net/pkg/fontawesome5 for full list of symbols.

% Change the page layout if you need to
\geometry{left=1.25cm,right=1.25cm,top=1.5cm,bottom=1.5cm,columnsep=1.2cm}

% The paracol package lets you typeset columns of text in parallel
\usepackage{paracol}


% Change the font if you want to, depending on whether
% you're using pdflatex or xelatex/lualatex
\ifxetexorluatex
  % If using xelatex or lualatex:
  \setmainfont{Lato}
\else
  % If using pdflatex:
  \usepackage[default]{lato}
\fi

% Change the colours if you want to
\definecolor{SlateGrey}{HTML}{2E2E2E}
\definecolor{LightGrey}{HTML}{666666}
\definecolor{Salmon}{HTML}{FA8072}
\definecolor{Aquamarine}{HTML}{088F8F}
\definecolor{TealBlue}{HTML}{008080}
%\usepackage[dvipsnames]{xcolor}
\colorlet{heading}{TealBlue}
\colorlet{accent}{Salmon}
\colorlet{emphasis}{Aquamarine}
\colorlet{body}{LightGrey}

% Change some fonts, if necessary
% \renewcommand{\namefont}{\Huge\rmfamily\bfseries}
% \renewcommand{\personalinfofont}{\footnotesize}
% \renewcommand{\cvsectionfont}{\LARGE\rmfamily\bfseries}
% \renewcommand{\cvsubsectionfont}{\large\bfseries}

% Change the bullets for itemize and rating marker
% for \cvskill if you want to
\renewcommand{\itemmarker}{{\small\textbullet}}
\renewcommand{\ratingmarker}{\faCircle}

%% Use (and optionally edit if necessary) this .tex if you
%% want to use an author-year reference style like APA(6)
%% for your publication list
% When using APA6 if you need more author names to be listed
% because you're e.g. the 12th author, add apamaxprtauth=12
\usepackage[backend=biber,style=apa6,sorting=ydnt]{biblatex}
\defbibheading{pubtype}{\cvsubsection{#1}}
\renewcommand{\bibsetup}{\vspace*{-\baselineskip}}
\AtEveryBibitem{\makebox[\bibhang][l]{\itemmarker}}
\setlength{\bibitemsep}{0.25\baselineskip}
\setlength{\bibhang}{1.25em}


%% Use (and optionally edit if necessary) this .tex if you
%% want an originally numerical reference style like IEEE
%% for your publication list
% \usepackage[backend=biber,style=ieee,sorting=ydnt]{biblatex}
%% For removing numbering entirely when using a numeric style
\setlength{\bibhang}{1.25em}
\DeclareFieldFormat{labelnumberwidth}{\makebox[\bibhang][l]{\itemmarker}}
\setlength{\biblabelsep}{0pt}
\defbibheading{pubtype}{\cvsubsection{#1}}
\renewcommand{\bibsetup}{\vspace*{-\baselineskip}}


%% sample.bib contains your publications
\addbibresource{sample.bib}

\begin{document}
\name{ZACHARY SHELTON}
\tagline{\textcolor{Aquamarine}{}}
% Cropped to square from https://en.wikipedia.org/wiki/Marissa_Mayer#/media/File:Marissa_Mayer_May_2014_(cropped).jpg, CC-BY 2.0
%% You can add multiple photos on the left or right
\photoR{3.5cm}{IMG_0577(1).jpg}
%\photoL{2cm}{Yacht_High,Suitcase_High}
\personalinfo{%
  % Not all of these are required!
  % You can add your own with \printinfo{symbol}{detail}
  \email{zacharshel10@gmail.com}
  \phone{+45 52 78 13 49}\\
  \location{Copenhagen, Demark}
    \bigskip
    \\
   Data engineer and scientist reinvent myself in a customer oriented position in a place or with a team I who shares my passion.
%   
%   \orcid{0000-0000-0000-0000} % Obviously making this up too.
  %% You can add your own arbitrary detail with
  %% \printinfo{symbol}{detail}[optional hyperlink prefix]
  % \printinfo{\faPaw}{Hey ho!}
  %% Or you can declare your own field with
  %% \NewInfoFiled{fieldname}{symbol}[optional hyperlink prefix] and use it:
  % \NewInfoField{gitlab}{\faGitlab}[https://gitlab.com/]
  % \gitlab{your_id}
	%%
  %% For services and platforms like Mastodon where there isn't a
  %% straightforward relation between the user ID/nickname and the hyperlink,
  %% you can use \printinfo directly e.g.
  %% But if you absolutely want to create new dedicated info fields for
  %% such platforms, then use \NewInfoField* with a star:
  % \NewInfoField*{mastodon}{\faMastodon}
  %% then you can use \mastodon, with TWO arguments where the 2nd argument is
  %% the full hyperlink.
  % \mastodon{@username@instance}{https://instance.url/@username}
}
\textbf{Authorized to Work in Denmark}\\ 
\smallskip
\makecvheader


%% Depending on your tastes, you may want to make fonts of itemize environments slightly smaller
\AtBeginEnvironment{itemize}{\small}

%% Set the left/right column width ratio to 6:4.
\columnratio{0.6}

% Start a 2-column paracol. Both the left and right columns will automatically
% break across pages if things get too long.
\begin{paracol}{2}


\cvsection{Work Experience}
\cvevent{Data Engineer}{Uptimus}{Aug 2024 - Dec 2024}{Copenhagen, DK}
\begin{itemize}
    \item Collaborated effectively with team members to maintain smooth operations, manage high-volume tasks, and uphold quality standards for beverages and cleanliness.
    \item Demonstrated strong attention to detail in order accuracy, cash handling, and inventory management, contributing to efficient daily operations.
    \item Adapted quickly to dynamic settings, proactively learning new skills and prioritizing tasks to deliver consistent results during peak hours
\end{itemize}
\cvevent{Student Research Coordinator}{Wolfram Research Inc.}{Jan 2020 - Feb 2024}{Copenhagen, DK (Remote)}
\begin{itemize}
    \item    Managed and mentored a diverse group of individuals, fostering collaboration and ensuring smooth operations in a fast-paced, team-oriented environment.

    \item    Provided personalized guidance and support to team members, helping them navigate tasks and improve efficiency in daily operations.

    \item    Trained individuals on best practices for workflow management, quality control, and maintaining high standards in a dynamic setting.

    \item    Adapted to changing priorities by coordinating tasks effectively, ensuring seamless operations during high-demand periods and special events.

    \item    Communicated clearly with individuals from diverse backgrounds, promoting a positive and inclusive team environment.

    Analyzed feedback and implemented actionable insights to improve processes and enhance overall team performance.
\end{itemize}

\cvsection{Education}
\cvevent{MSc Physics, Focus Area High Energy Physics}{Florida Institute of Technology}{Jan 2020 - Dec 2021}{Melbourne, United States}



\textcolor{Aquamarine}{Using Machine Learning to identify Top Quarks in CMS (Feb 2021 - Sept 2021)}
\begin{itemize}
    \item Used Python library XgBoost to identify and match top quark pairs via charge and geometric characteristics of  hadronic decay jets.
\end{itemize}
\cvevent{BSc Physics and STEM Education}{Florida Institute of Technology}{August 2015 - December 2019}{Melbourne, United States}
\textcolor{Aquamarine}{Educational Cosmic Muon Detector - BSc Thesis (January 2019 - December 2019)}
\switchcolumn

\cvsection{Skills \& Competencies}

\cvtag{Communication}
\cvtag{Cleanliness}\\
\cvtag{Customer Service}
\cvtag{Microsoft Office}
\cvtag{Mathematics}
\cvtag{Food Safety}\\
\cvtag{Shift Planning}
\cvtag{Time Management}\\
\cvtag{Point of Sale}
\cvtag{Planning}
\cvtag{Attention to Detail}

\cvsection{Professional\\\smallskip Interests}


\cvtag{Particle Physics}
\cvtag{Data Science}\\
\cvtag{Artificial Intelligence}
\cvtag{Green Energy}
\cvtag{Data Visualization}


\medskip




\cvsection{Languages}
\cvtag{English(Native Speaker)} \\
\cvtag{Danish (PD 2.5)}

% \bigskip
% \bigskip
% \bigskip
% \bigskip
% \bigskip
% \bigskip
% \bigskip
% \bigskip
% \bigskip
% \bigskip
% \bigskip


\cvsection{Hobbies and \\\smallskip Interests}
\cvtag{Snowboarding}
\cvtag{Hiking}
\cvtag{Camping} \\
\cvtag{Gaming}
\cvtag{Electronics prototyping}

\cvsection{Work Experience (Cont)}


\cvevent{Undergraduate Research Assistant}{Florida Institute of Technology}{May 2016-Dec 2019  }{Melbourne, USA}
\begin{itemize}
\item Work under Associate Professor Ph.D. Francisco Yumiceva.
\item Design a Print Circuit Board (PCB) (Using Eagle CAD) for use in particle physics labs including FermiLab, and the CMS detector at CERN.
\item Spent 3 months working on multinational team testing CMS equipment at Fermilab.
\item Maintained team python software base while at Fermi-Lab.
\item Use ROOT7 analysis framework and Python to analyze CERN Analysis Object Data.
\end{itemize}

\end{paracol}

\end{document}
