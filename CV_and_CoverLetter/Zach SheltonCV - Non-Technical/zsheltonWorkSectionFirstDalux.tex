%%%%%%%%%%%%%%%%%
% This is an example CV created using altacv.cls (v1.6.4, 13 Nov 2021) written by
% LianTze Lim (liantze@gmail.com), based on the
% Cv created by BusinessInsider at http://www.businessinsider.my/a-sample-resume-for-marissa-mayer-2016-7/?r=US&IR=T
%
%% It may be distributed and/or modified under the
%% conditions of the LaTeX Project Public License, either version 1.3
%% of this license or (at your option) any later version.
%% The latest version of this license is in
%%    http://www.latex-project.org/lppl.txt
%% and version 1.3 or later is part of all distributions of LaTeX
%% version 2003/12/01 or later.
%%%%%%%%%%%%%%%%

%% Use the "normalphoto" option if you want a normal photo instead of cropped to a circle
% \documentclass[10pt,a4paper,normalphoto]{altacv}

\documentclass[10pt,a4paper,ragged2e,withhyper]{altacv}

%% AltaCV uses the fontawesome5 package.
%% See http://texdoc.net/pkg/fontawesome5 for full list of symbols.

% Change the page layout if you need to
\geometry{left=1.25cm,right=1.25cm,top=1.5cm,bottom=1.5cm,columnsep=1.2cm}

% The paracol package lets you typeset columns of text in parallel
\usepackage{paracol}


% Change the font if you want to, depending on whether
% you're using pdflatex or xelatex/lualatex
\ifxetexorluatex
  % If using xelatex or lualatex:
  \setmainfont{Lato}
\else
  % If using pdflatex:
  \usepackage[default]{lato}
\fi

% Change the colours if you want to
\definecolor{SlateGrey}{HTML}{2E2E2E}
\definecolor{LightGrey}{HTML}{666666}
\definecolor{Salmon}{HTML}{FA8072}
\definecolor{Aquamarine}{HTML}{088F8F}
\definecolor{TealBlue}{HTML}{008080}
%\usepackage[dvipsnames]{xcolor}
\colorlet{heading}{TealBlue}
\colorlet{accent}{Salmon}
\colorlet{emphasis}{Aquamarine}
\colorlet{body}{LightGrey}

% Change some fonts, if necessary
% \renewcommand{\namefont}{\Huge\rmfamily\bfseries}
% \renewcommand{\personalinfofont}{\footnotesize}
% \renewcommand{\cvsectionfont}{\LARGE\rmfamily\bfseries}
% \renewcommand{\cvsubsectionfont}{\large\bfseries}

% Change the bullets for itemize and rating marker
% for \cvskill if you want to
\renewcommand{\itemmarker}{{\small\textbullet}}
\renewcommand{\ratingmarker}{\faCircle}

%% Use (and optionally edit if necessary) this .tex if you
%% want to use an author-year reference style like APA(6)
%% for your publication list
\input{pubs-authoryear}

%% Use (and optionally edit if necessary) this .tex if you
%% want an originally numerical reference style like IEEE
%% for your publication list
% \input{pubs-num}

%% sample.bib contains your publications
\addbibresource{sample.bib}
\begin{document}
\name{ZACHARY SHELTON}
\tagline{\textcolor{Aquamarine}{Data Scientist}}
% Cropped to square from https://en.wikipedia.org/wiki/Marissa_Mayer#/media/File:Marissa_Mayer_May_2014_(cropped).jpg, CC-BY 2.0
%% You can add multiple photos on the left or right
\photoR{3.5cm}{USEME.jpeg}
%\photoL{2cm}{Yacht_High,Suitcase_High}
\personalinfo{%
  % Not all of these are required!
  % You can add your own with \printinfo{symbol}{detail}
  \email{zacharshel10@gmail.com}
  \phone{+45 52 78 13 49}\\
  \location{Copenhagen, Demark}
    \linkedin{zshelton}
    \bigskip
    \\
     Data scientist with a passion for scientific discovery and communication. I leverage my skills in physics and data analysis to conduct practical research in data science and AI.
% MY COOL PHYSICS WRITTEN INTRO
%Physicist with a passion for scientific discovery and communication. I leverage my skills in experimental optics and software development to design and conduct impactful research in the field of optics.
%   \github{github.com/mmayer} % I'm just making this up though.
%   \orcid{0000-0000-0000-0000} % Obviously making this up too.
  %% You can add your own arbitrary detail with
  %% \printinfo{symbol}{detail}[optional hyperlink prefix]
  % \printinfo{\faPaw}{Hey ho!}
  %% Or you can declare your own field with
  %% \NewInfoFiled{fieldname}{symbol}[optional hyperlink prefix] and use it:
  % \NewInfoField{gitlab}{\faGitlab}[https://gitlab.com/]
  % \gitlab{your_id}
	%%
  %% For services and platforms like Mastodon where there isn't a
  %% straightforward relation between the user ID/nickname and the hyperlink,
  %% you can use \printinfo directly e.g.
  %% But if you absolutely want to create new dedicated info fields for
  %% such platforms, then use \NewInfoField* with a star:
  % \NewInfoField*{mastodon}{\faMastodon}
  %% then you can use \mastodon, with TWO arguments where the 2nd argument is
  %% the full hyperlink.
  % \mastodon{@username@instance}{https://instance.url/@username}
}
\textbf{Authorized to Work in Denmark}\\ 
\smallskip
\makecvheader


%% Depending on your tastes, you may want to make fonts of itemize environments slightly smaller
\AtBeginEnvironment{itemize}{\small}

%% Set the left/right column width ratio to 6:4.
\columnratio{0.6}

% Start a 2-column paracol. Both the left and right columns will automatically
% break across pages if things get too long.
\begin{paracol}{2}



\cvsection{Work Experience}

\cvevent{Student Research Coordinator}{Wolfram Research Inc.}{Jan 2021 - Feb 2024}{Champaign, United States/Remote}
\begin{itemize}
    \item Manage Wolfram's Student Ambassador Program.
    \item Provide on-campus support for ambassadors at Universities teaching Wolfram Language.
    \item Advise and assist students in Wolfram Language in the frame of the students research or academics.
    \item Train students in creating and debugging machine learning algorithms, neural networks, time-series modelling and feature extraction.
    \item Grow my skills in database management(SQLite) and numerical analysis to teach students from high school till graduate levels.
    \item Conduct qualitative textual and sentiment analysis of Wolfram's social media and coverage in other media.
    \item Explore Wolfram's Quantum Framework tools, Neural Net Library, and data analysis tools.
    \item Worked as Technical Outreach Intern from Jan 2021 to Mar 2021, after which assuming above role.
\end{itemize}

\medskip
\cvevent{Undergraduate Research Assistant}{Florida Institute of Technology}{May 2016 - Dec 2019}{Melbourne, United States}
\begin{itemize}
\item Work under Associate Professor Ph.D. Francisco Yumiceva.
\item Design a Print Circuit Board (PCB) (Using Eagle CAD) for use in particle physics labs including FermiLab, and the CMS detector at CERN.
\item PCB is responsible for calibration of QIE charge injectors by simulating calorimeter responses.
\item Spent 3 months working on multinational team testing CMS equipment at Fermilab.
\item Maintained team python software base while at Fermi-Lab.
\item Use ROOT7 analysis framework and Python to analyze CERN Analysis Object Data.
\item Utilized Electronic Measurement techniques and SMD rework for cosmic ray observation.
\item Training new members of team to continue team’s project.
\end{itemize}
\cvevent{Electrical Engineering Intern}{Jaycon Systems}{May 2019 - Aug 2019}{Melbourne, United States}
\begin{itemize}
\item Worked under lead Electrical Engineer communicating with clients.
\item Designed and created prototype boards for consumer electronics.
\item Learned and completed assembly process of for large batch PCB boards.
\item Gained proficiency soldering and other methods for modifying PCB boards.
\end{itemize}
\medskip
\newpage

\cvsection{Relevant Academic Projects}
\textcolor{Aquamarine}{Educational Cosmic Muon Detector - BSc Thesis (January 2019 - December 2019)}
\smallskip
\begin{itemize}
    \item Assembled a simple cosmic ray detector for use in undergraduate physics courses made from a recycled scintillator. 
    \item Spectrum analysis of scintillating materials from FermiLab.
    \item Created course material to accompany the device.
\end{itemize}
\smallskip
\textcolor{Aquamarine}{Using Machine Learning to identify Top Quarks in CMS (February 2021 - September 2021)}
\smallskip
\begin{itemize}
    \item Used Python library XgBoost to identify and match top quark pairs via charge and geometric characteristics of  hadronic decay jets.
    \item Compared geometric and spatial orientation of 3 or more jets to identify most likely trio quarks from a top decay.
    \item Determined that boosted  decision trees combined with data regression methods would be the most effective method for diverse datasets.
\end{itemize}
\smallskip
\switchcolumn


\cvsection{Education}
\cveventtwo{MSc Physics, Focus Area High Energy Physics}{Florida Institute of Technology}{Jan 2020 - Dec 2021}{Melbourne, United States}
%%%
% \textcolor{Aquamarine}{Using Machine Learning to identify Top Quarks in CMS (February 2021 - September 2021)}
% \begin{itemize}
%     \item Used Python library XgBoost to identify and match top quark pairs via charge and geometric characteristics of  hadronic decay jets.
%     \item Compared geometric and spatial orientation of 3 or more jets to identify most likely trio quarks from a top decay.
%     \item Determined that boosted  decision trees combined with data regression methods would be the most effective method for diverse datasets.
% \end{itemize}
\cveventtwo{BSc Physics}{Florida Institute of Technology}{Aug 2015 - Dec 2019}{ Melbourne, United States}
% \textcolor{Aquamarine}{Educational Cosmic Muon Detector - BSc Thesis (January 2018 - December 2018)}
% \begin{itemize}
%     \item Assembled a simple cosmic ray detector for use in undergraduate physics courses made from a recycled scintillator. 
%     \item Spectrum analysis of scintillating materials from FermiLab.
%     \item Created course material to accompany the device.
% \end{itemize}
%%%
\cveventtwo{BSc STEM Education}{Florida Institute of Technology}{Aug 2015 - Dec 2019}{Melbourne, United States}

\cvsection{Computer languages \\\smallskip \& Techincal Skills}
\cvtag{Experimental Design}
\cvtag{Python}
\cvtag{PyTorch}
\cvtag{HTML}
\cvtag{Git}
\cvtag{NLP}
\cvtag{Tensorflow/Keras}\\
\cvtag{Text Analysis}
\cvtag{Time-Series Modelling}
\cvtag{SQLite}
\cvtag{Numerical Analysis}\\
\cvtag{Bayesian Analysis}
\cvtag{CENTOS7/Linux}
\cvtag{Pandas}
\cvtag{Keras}
\cvtag{XgBoost}
\cvtag{scikit}
\cvtag{Wolfram Language}
\cvtag{LaTeX}

\cvtag{Microsoft Excel, Word, PowerPoint}\\
\cvtag{C++}
\cvtag{EagleCAD}

\cvsection{Professional\\\smallskip Interests}


\cvtag{Data Science}\\
\cvtag{Neural Networks}
\cvtag{Artificial Intelligence}\\
\cvtag{Solid State Physics}\
\cvtag{Scientific Literacy}
\cvtag{Optics}
\cvtag{Engineering Physics}
\cvtag{High Energy Physics}
\cvtag{Optics}
\cvtag{Science Education}
\cvtag{Public Policy}







\medskip




\cvsection{Languages}
\cvtag{English (Native Speaker)}
\newline
\cvtag{Danish (Intermediate Proficiency)}

% \bigskip
% \bigskip
% \bigskip
% \bigskip
% \bigskip
% \bigskip
% \bigskip
% \bigskip
% \bigskip
% \bigskip
% \bigskip
\newpage
\cvsection{Relevant Coursework}
\begin{itemize}
    \item {\normalsize Data Analysis Methods}
    \item {\normalsize Electromagnetic Theory 2}
    \item {\normalsize Science and Technical Communications}
    \item {\normalsize Mathematical Methods in Science and \\Engineering}
    \item {\normalsize Quantum Mechanics 2}
    \item {\normalsize Statistics in High Energy Physics}
\end{itemize}


\cvsection{Hobbies and \\\smallskip Interests}
\cvtag{Snowboarding}
\cvtag{Hiking}
\cvtag{Camping} \\
\cvtag{Gaming}
\cvtag{Electronics prototyping}

\end{paracol}

\end{document}
