%-------------------------
% Entry-level Cover-letter Template in LaTeX
% Made to go with "Entry-level Resume in laTeX" - here
% Version - v1.0
% Last Edits - October 5, 2021
% Author : Jayesh Sanwal
% Reach out to me on LinkedIn(/in/jsanwal), with any suggestions, ideas, issues, etc.
%------------------------
% !TEX program = xelatex

%%%% Define Document type
\documentclass[11pt,a4]{article}

%%%% Include Packages
\usepackage{latexsym}
\usepackage[empty]{fullpage}
\usepackage{titlesec}
 \usepackage{marvosym}
\usepackage[usenames,dvipsnames]{color}
\usepackage{verbatim}
\usepackage[hidelinks]{hyperref}
\usepackage{fancyhdr}
\usepackage{multicol}
\usepackage{hyperref}
\usepackage{csquotes}
\usepackage{tabularx}
\hypersetup{colorlinks=true,urlcolor=black}
\usepackage[11pt]{moresize}
\usepackage{setspace}
\usepackage{fontspec}
\usepackage[inline]{enumitem}
\usepackage{ragged2e}
\usepackage{anyfontsize}

%%%% Set Margins
\usepackage[margin=1cm]{geometry}

%%%% Set Fonts
\setmainfont[
BoldFont=SourceSansPro-Semibold.otf,
ItalicFont=SourceSansPro-RegularIt.otf
]{SourceSansPro-Regular.otf}

%%%% Set Page Style
\pagestyle{fancy}
\fancyhf{} 
\fancyfoot{}
\renewcommand{\headrulewidth}{0pt}
\renewcommand{\footrulewidth}{0pt}

%%%% Set URL Style
\urlstyle{same}

%%%% Set Indentation
\raggedbottom
\raggedright
\setlength{\tabcolsep}{0in}

%%%% Set Secondary Color, Page Number Color, Footer Text
\definecolor{UI_blue}{RGB}{32, 64, 151}
\definecolor{HF_color}{RGB}{179, 179, 179}

%%%% Set Heading Format
\titleformat{\section}{
\color{UI_blue} \scshape \raggedright \large 
}{}{0em}{}[\vspace{-0.7cm} \hrulefill \vspace{-0.2cm}]
%%%%%%% --------------------------------------------------------------------------------------
%%%%%%% --------------------------------------------------------------------------------------
%%%%%%%  END OF "DO NOT TOUCH" REGION
%%%%%%% --------------------------------------------------------------------------------------
%%%%%%% --------------------------------------------------------------------------------------



\begin{document}
%%%%%%% --------------------------------------------------------------------------------------
%%%%%%%  HEADER
%%%%%%% --------------------------------------------------------------------------------------
\begin{center}
    \begin{minipage}[b]{0.24\textwidth}
            \large 52 78 13 49 \\
            \large \href{mailto:zacharshel10@gmail.com}{zacharshel10@gmail.com} 
    \end{minipage}% 
    \begin{minipage}[b]{0.5\textwidth}
            \centering
            {\Huge Zach Shelton} \\ %
            \vspace{0.1cm}
            {\color{UI_blue} \Large{software Engineer}} \\
    \end{minipage}% 
    \begin{minipage}[b]{0.24\textwidth}
            \flushright \large
            {\href{https://www.linkedin.com/in/zshelton/}{linkedin.com/in/zshelton/} } \\

    \end{minipage}   
    
\vspace{-0.15cm} 
{\color{UI_blue} \hrulefill}
\end{center}

\justify
\setlength{\parindent}{0pt}
\setlength{\parskip}{12pt}
\vspace{0.1cm}

%%%%%%% --------------------------------------------------------------------------------------
%%%%%%%  First 2 Lines
%%%%%%% --------------------------------------------------------------------------------------

Date: \today \par \vspace{-0.1cm}

To Professor Anders Bjorholm Dahl,\

I am writing to express my strong interest in joining DTU Compute. My academic background in experimental particle physics, combined with hands-on experience at a data-driven startup in Denmark, has prepared me to contribute effectively to a team of curious, driven professionals. With expertise in research and software development, I am ready to make an immediate impact at DTU.

During my studies, I collaborated with a multinational team on the Compact Muon Solenoid (CMS) Experiment at CERN, where I developed machine learning algorithms to analyze large-scale datasets. My work involved building predictive models to isolate specific decay events from kinematics data using neural networks, decision trees, and libraries like XGBoost, TensorFlow, and Keras. This combined experience in particle kinematics and ML algorithms makes me well-suited to contribute to QUAITOM.

At Wolfram Research, I taught Wolfram Language and mentored university students on data-intensive projects, including financial modeling, statistical analysis, and machine learning applications. These projects spanned GIS, mathematics, physics, and chemistry, sharpening my ability to adapt to new domains and communicate insights clearly. I also managed student projects using SCRUM-like methodologies, ensuring timelines and accountability—demonstrating my ability to thrive in large teams and quickly take on new tasks in support of research.

Most recently, as a Data Engineer at Uptimus, a Danish startup focused on last-mile delivery optimization, I designed ETL pipelines and data integration solutions using GIT, Docker, Azure, PostgreSQL, and Python. By processing real-time data from delivery vehicles across Sjælland, I gained hands-on experience in workflow optimization and version control—critical skills for collaborative development.

Across all my roles, I have worked with and adjacent to HPC clusters while applying physics-based data analysis. I can provide references, including my supervisor at Uptimus, who can speak to my technical skills and work ethic. Please feel free to contact me via phone or email to discuss my application or schedule an interview.
%%%
%%%%%%% --------------------------------------------------------------------------------------
%%%%%%%  SIGNATURE
%%%%%%% --------------------------------------------------------------------------------------

\vspace{0.5cm}
\raggedright
Respectfully, \\ Zach Shelton \\ +45 52 78 13 49 \\ 
\href{mailto:zshelton1997@gmail.com}{zacharshel10@gmail.com} 

\end{document}


