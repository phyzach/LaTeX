%-------------------------
% Entry-level Cover-letter Template in LaTeX
% Made to go with "Entry-level Resume in laTeX" - here
% Version - v1.0
% Last Edits - October 5, 2021
% Author : Jayesh Sanwal
% Reach out to me on LinkedIn(/in/jsanwal), with any suggestions, ideas, issues, etc.
%------------------------
% !TEX program = xelatex

%%%% Define Document type
\documentclass[11pt,a4]{article}

%%%% Include Packages
\usepackage{latexsym}
\usepackage[empty]{fullpage}
\usepackage{titlesec}
 \usepackage{marvosym}
\usepackage[usenames,dvipsnames]{color}
\usepackage{verbatim}
\usepackage[hidelinks]{hyperref}
\usepackage{fancyhdr}
\usepackage{multicol}
\usepackage{hyperref}
\usepackage{csquotes}
\usepackage{tabularx}
\hypersetup{colorlinks=true,urlcolor=black}
\usepackage[11pt]{moresize}
\usepackage{setspace}
\usepackage{fontspec}
\usepackage[inline]{enumitem}
\usepackage{ragged2e}
\usepackage{anyfontsize}

%%%% Set Margins
\usepackage[margin=1cm]{geometry}

%%%% Set Fonts
\setmainfont[
BoldFont=SourceSansPro-Semibold.otf,
ItalicFont=SourceSansPro-RegularIt.otf
]{SourceSansPro-Regular.otf}

%%%% Set Page Style
\pagestyle{fancy}
\fancyhf{} 
\fancyfoot{}
\renewcommand{\headrulewidth}{0pt}
\renewcommand{\footrulewidth}{0pt}

%%%% Set URL Style
\urlstyle{same}

%%%% Set Indentation
\raggedbottom
\raggedright
\setlength{\tabcolsep}{0in}

%%%% Set Secondary Color, Page Number Color, Footer Text
\definecolor{UI_blue}{RGB}{32, 64, 151}
\definecolor{HF_color}{RGB}{179, 179, 179}

%%%% Set Heading Format
\titleformat{\section}{
\color{UI_blue} \scshape \raggedright \large 
}{}{0em}{}[\vspace{-0.7cm} \hrulefill \vspace{-0.2cm}]
%%%%%%% --------------------------------------------------------------------------------------
%%%%%%% --------------------------------------------------------------------------------------
%%%%%%%  END OF "DO NOT TOUCH" REGION
%%%%%%% --------------------------------------------------------------------------------------
%%%%%%% --------------------------------------------------------------------------------------



\begin{document}
%%%%%%% --------------------------------------------------------------------------------------
%%%%%%%  HEADER
%%%%%%% --------------------------------------------------------------------------------------
\begin{center}
    \begin{minipage}[b]{0.24\textwidth}
            \large 52 78 13 49 \\
            \large \href{mailto:zacharshel10@gmail.com}{zacharshel10@gmail.com} 
    \end{minipage}% 
    \begin{minipage}[b]{0.5\textwidth}
            \centering
            {\Huge Zach Shelton} \\ %
            \vspace{0.1cm}
            {\color{UI_blue} \Large{software Engineer}} \\
    \end{minipage}% 
    \begin{minipage}[b]{0.24\textwidth}
            \flushright \large
            {\href{https://www.linkedin.com/in/zshelton/}{linkedin.com/in/zshelton/} } \\

    \end{minipage}   
    
\vspace{-0.15cm} 
{\color{UI_blue} \hrulefill}
\end{center}

\justify
\setlength{\parindent}{0pt}
\setlength{\parskip}{12pt}
\vspace{0.1cm}

%%%%%%% --------------------------------------------------------------------------------------
%%%%%%%  First 2 Lines
%%%%%%% --------------------------------------------------------------------------------------

Date: \today \par \vspace{-0.1cm}

To whom it may concern,

I am writing to express my strong interest in joining DTU's department of Physics working on quantum key distribution systems. My academic background in experimental particle physics, combined with hands-on experience at a data-driven startup in Denmark, has prepared me to contribute effectively to a team of curious, driven, and like-minded professionals. I have ample academic research and software developement experience at my previous positions and am ready to contribute to a team on the cutting edge science and technology.

During my studies, I collaborated with a multinational team on the Compact Muon Solenoid (CMS) Experiment at CERN, where I developed machine learning algorithms to analyze large-scale datasets. My work involved creating predictive models to isolate specific decay events from kinematics data using neural networks, decision trees, and libraries such as XGBoost, TensorFlow, and Keras. I also designed electronics and supporting software to test the functioning of PCBs, I participated in the total process of design, testing and implementation, these experiences not only honed my Python skills, but also strengthened my ability to handle the prototyping, testing and QA required that is a part of software development and research.

In my role at Wolfram Research, I taught Wolfram Language and guided university students on data-intensive projects, including financial modeling, statistical analysis, and machine learning applications. These projects spanned diverse fields such as GIS, mathematics, physics, and chemistry, giving me the adaptability to quickly learn new domains and communicate insights. I treated the students' project in a way similar to a SCRUM attempting to maintain proper timelines and accountability. At Wolfram, I also spent time utilizing Wolfram's Quantum Computing Framework to explore quantum systems and accurately simulate them.

Most recently, as a Data Engineer at Uptimus, a Danish startup focused on last-mile delivery optimization, I designed ETL pipelines and data integration solutions using GIT. My work involved preparing production-ready scripts, leveraging tools like Docker, Azure, and PostgreSQL, and applying geospatial analysis with QGIS and Python libraries. With the data streamed from delivery vehicles throughout Sjælland, I developed a keen ability to think on my feet and learn important software tools necessary for working and contributing in an academic environment.

At DTU, I look forward to contributing to dynamic data streaming pipelines while deepening my expertise in Python and exploring new techniques. I finished my Danish classes at level PD 2.5, and I am able to write professional emails and communications. I have attached references, including my supervisor at Uptimus, who can speak to my technical abilities and work ethic. Please feel free to contact me via phone or email to discuss my application or arrange an interview.

Thank you for considering my application. I am excited about the opportunity to learn and contribute at DTU!
%%%
%%%%%%% --------------------------------------------------------------------------------------
%%%%%%%  SIGNATURE
%%%%%%% --------------------------------------------------------------------------------------

\vspace{0.5cm}
\raggedright
Respectfully, \\ Zach Shelton \\ +45 52 78 13 49 \\ 
\href{mailto:zshelton1997@gmail.com}{zacharshel10@gmail.com} 

\end{document}



%I am most proud of my work in Experimental Particle Physics with collision data from CERN. I participated in a multi-national team and create truly interesting and great work in machine learning in physics.

Secondly, I am glad I spent time teaching and advising students using Wolfram Language. The projects varied from introductory CS, to PhD research and big data projects.